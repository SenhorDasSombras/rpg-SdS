%!TeX root=../../rules.tex
\chapter{Guerreiro}%
\label{cha:guerreiro}
\begin{multicols}{2}

\section*{Características de Classe}%

Como um guerreiro, você adquire as seguintes características de classe.

\subsubsection{Pontos de Vida}%

\noindent\textbf{Dado de Vida}: 1d10 por nível de guerreiro \nl
\textbf{Pontos de Vida no 1º Nível:} 10 + seu modificador de Constituição. \nl
\textbf{Pontos de Vida nos Níveis Seguintes:} 1d10 (ou 6) + seu modificador de
Constituição por nível de guerreiro após o 1º.

\subsubsection{Proficiências}%

\noindent\textbf{Armaduras:} Todas as armaduras, escudos \nl
\textbf{Armas:} Armas simples, armas marciais \nl
\textbf{Ferramentas:} Nenhuma \jump
\textbf{Testes de Resistência}: Força, Constituição \nl
\textbf{Pericias:} Escolha duas entre Acrobacia, Adestrar Animais, Atletismo,
História, Intuição, Intimidação, Percepção e Sobrevivência.

\subsubsection{Equipamento}%

Você começa com o seguinte equipamento, além do equipamento concedido pelo seu
antecedente.
\begin{itemize}
    \item (a) uma cota de malha ou (b) um gibão de peles, um arco longo e 20
        flechas;
    \item (a) uma arma marcial e um escudo ou (b) duas armas marciais;
    \item (a) uma besta leve e 20 virotes ou (b) dois machados de arremesso;
    \item (a) um pacote de aventureiro ou (b) um pacote de explorador.
\end{itemize}

\section*{Conjuração}%

A tabela a seguir mostra em quais níveis o guerreiro tem acesso às magias de
cada ciclo:

\begin{center}
\begin{tabular}{|||c||c|||}
    \hline
    \textbf{Nível} & \textbf{Ciclo} \\
    \hline
    2 & 1º \\
    \hline
    5 & 2º \\
    \hline
    8 & 3º \\
    \hline
    11 & 4º \\
    \hline
    14 & 5º \\
    \hline
    17 & 6º \\
    \hline
    20 & 7º \\
    \hline
\end{tabular}
\end{center}

\subsubsection*{Habilidade de Conjuração}%

Força é a sua habilidade para você conjurar suas magias de guerreiro, pois
guerreiros impõem sua vontade no mundo natural através de sua força e
treinamento corporal. Você usa sua Força sempre que alguma magia se referir à
sua habilidade de conjurar magias. Além disso, você usa o seu modificador de
Força para definir a CD dos testes de resistência para as magias de guerreiro
que você conjura e quando você realiza uma jogada de ataque com uma magia.

\begin{center}
\textbf{CD para suas magias} = 8 + bônus de proficiência + seu modificador de
Força. \nl

\textbf{Modificador de ataque de magia} = seu bônus de proficiência + seu
modificador de Força
\end{center}

\section*{Estilo de Luta}%

Você adota um estilo de combate particular que será sua especialidade. Escolha
uma das opções a seguir. Você não pode escolher o mesmo Estilo de Combate mais
de uma vez, mesmo se puder escolher novamente.

\subsubsection{Arremesso de Armas}%

Você pode sacar uma arma que possua a propriedade de arremesso como parte de sua
ação de ataque com essa arma.

Adicionalmente, quando você acerta um ataque à distância usando uma arma de
arremesso, você ganha um bônus de +2 na jogada de dano.

\subsubsection{Ataque Desarmado}%

Seus ataques desarmados podem provocar dano contundente igual a 1d6 + seu
modificador de Força em caso de acerto. Se você não estiver empunhando nenhuma
arma ou escudo ao realizar a jogada de ataque, esse 1d6 se transforma em 1d8.

No começo de cada um de seus turnos, você pode causar 1d4 de dano contundente a
uma criatura agarrada por você.

\subsubsection{Combate com Armas Grandes}%

Quando você rolar um $1$ ou $2$ num dado de dano de um ataque com uma arma
corpo-a-corpo que você esteja empunhando com duas mãos, você pode rolar o dado
novamente e usar a nova rolagem. A arma deve ter a propriedade duas mãos ou
versátil para ganhar esse benefício.

\subsubsection{Combate com Duas Armas}%

Quando você estiver engajado em uma luta com duas armas, você pode adicionar o
seu modificador de habilidade de dano na jogada de dano de seu segundo ataque.

\subsubsection{Defesa}%

Enquanto estiver usando armadura, você ganha +1 de bônus em sua CA.

\subsubsection{Duelismo}%

Quando você empunhar uma arma de ataque corpo-a-corpo em uma mão e nenhuma outra
arma, você ganha +2 de bônus nas jogadas de dano com essa arma.

\subsubsection{Interceptador}%

Quando uma criatura que você possa ver acerta um alvo que não seja você, a até
1,5m de distância de você com um ataque, você pode usar sua reação para reduzir
o dano recebido por esse alvo em 1d10 + seu bônus de proficiência (até um mínimo
de 0 de dano). Você deve estar empunhando um escudo ou uma arma simples ou
marcial para usar essa reação.

\subsubsection{Luta as Cegas}%

Você tem percepção às cegas com um alcance de 3m. Dentro desse alcance, você
pode efetivamente ver qualquer coisa que não esteja sob cobertura total, mesmo
se você estiver cego ou na escuridão. Além disso, você pode ver uma criatura
invisível nessa área, a menos que a criatura se esconda de você com sucesso.

\subsubsection{Proteção}%

Quando uma criatura que você possa ver atacar um alvo que esteja a até 1,5 metro
de você, você pode usar sua reação para impor desvantagem nas jogadas de ataque
da criatura. Você deve estar empunhando um escudo.

\subsubsection{Técnica Superior}%

Você aprende uma manobra a sua escolha dentre aquelas disponíveis para o
arquétipo do Mestre de Batalha. Se uma manobra que você utilizar exigir que o
alvo realize uma jogada de salvaguarda para resistir ao efeito dela, a CD da
salvaguarda será igual a 8 + seu bônus de proficiência + seu modificador de
Força ou Destreza (o que você preferir).

Você ganha um dado de superioridade, que é 1d6 (esse dado é adicionado a
quaisquer dados de superioridade que você tenha de outra fonte). Esse dado é
usado para abastecer suas manobras. Um dado de superioridade é gasto quando você
o utiliza. Você recupera os dados gastos quando você termina um descanso curto
ou longo.

\section*{Retomar o Fôlego}%

Você possui uma reserva de estamina e pode usá-la para proteger a si mesmo
contra danos. No seu turno, você pode usar uma ação bônus para recuperar pontos
de vida igual a 1d10 + seu nível de guerreiro.

Uma vez que você use essa característica, você precisa terminar um descanso
curto ou longo para usá-la de novo.

\section*{Surto de Ação}%

A partir do 2º nível, você pode forçar o seu limite para além do normal por um
momento. Durante o seu turno, você pode realizar uma ação adicional juntamente
com sua ação e possível ação bônus.

Uma vez que você use essa característica, você precisa terminar um descanso
curto ou longo para usá-la de novo.

A partir do 17º nível, você pode usá-la duas vezes antes do descanso, porém
somente uma vez por turno.

\section*{Arquétipo Marcial}%

No 3º nível, você escolhe um arquétipo o qual se esforçará para seguir as
técnicas e estilos de combate dele. Todos os arquétipos estão detalhados no
final da descrição da classe. O arquétipo confere a você características
especial no 3º nível e de novo nos 7º, 10º, 15º e 18º nível.

\section*{Incremento em Habilidade}%

Quando você atinge o 4º nível e novamente no 8º, 12º, 16º e 19º nível, você pode
ganha dois pontos de habilidades para distribuir entre suas habilidades. Por
padrão, você não pode elevar um valor de habilidade acima de 20 com essa
característica.

Opcionalmente, você pode escolher um talento seguindo as regras de talentos.

\section*{Ataque Extra}%

A partir do 5º nível, você pode atacar duas vezes, ao invés de uma, quando usar
a ação Atacar durante o seu turno.

O número de ataques aumenta para três quando você alcançar o 11º nível de
guerreiro e para 4 quando alcançar o 20º nível de guerreiro.

\section*{Indomável}%

A partir do 9º nível, você pode jogar de novo um teste de resistência que
falhou. Se o fizer, você deve usar o novo valor e não pode usar essa
característica de novo antes de terminar um descanso longo.

Você pode usar esta característica duas vezes entre descansos longos quando
chegar no 13º nível e três vezes entre descansos longos quando chegar no 17º
nível.

\section*{Arquétipos Marciais}%

Diferentes guerreiros escolhem diferentes caminhos para aperfeiçoar seu poder em
combate. O arquétipo marcial que você escolhe seguir reflete essa escolha.

\subsection*{Campeão}%

O arquétipo Campeão foca no desenvolvimento da pura força física acompanhada por
uma perfeição mortal.

Aqueles que trilham o caminho desse arquétipo combinam rigorosos treinamentos
com excelência física para desferir golpes devastadores.

\subsubsection{Crítico Aprimorado}%

A partir do 3º nível, seus ataques com armas adquirem uma margem de acerto
crítico de 19 a 20 nas jogadas de ataque.

\subsubsection{Atletismo Extraordinário}%

A partir do 7º nível, você adiciona metade de seu bônus de proficiência
(arredondado para cima) em qualquer teste de Força, Destreza ou Constituição que
você já não aplique seu bônus de proficiência.

Além disso, quando você fizer um salto longo com corrida, o alcance em metros
que poderá saltar aumenta em 0,3 vezes o seu modificador de Força.

\subsubsection{Estilo de Luta Adicional}%

No 10º nível, você pode escolher um segundo Estilo de Combate da sua
característica de classe.

\subsubsection{Crítico Superior}%

A partir do 15º nível, seus ataques com armas adquirem uma margem de acerto
crítico de 18 a 20 nas jogadas de ataque.

\subsubsection{Sobrevivente}%

No 18º nível, você alcança o topo da resiliência em batalha. No começo de cada
um de seus turnos, você recupera uma quantidade de pontos de vida igual a 5 +
seu modificador de Constituição se não estiver com mais que metade de seus
pontos de vida. Você não recebe esse benefício se estiver com 0 pontos de vida.

\subsection*{Mestre de Batalha}%

Aqueles que emulam o arquétipo de Mestre de Batalha empregam técnicas marciais
passadas de geração em geração. Para um Mestre de Batalha, o combate é um campo
acadêmico, as vezes, incluindo assuntos além da batalha, como forjaria e
caligrafia. Nem todo guerreiro absorve as lições de história, teoria e arte que
são refletidas no arquétipo de Mestre de Batalha, mas aqueles que conseguem,
tornam-se guerreiros bem-supridos de grande perícia e conhecimento.

\subsubsection{Superioridade em Combate}%

Quando você escolhe esse arquétipo, no 3° nível, você aprende manobras que são
abastecidas com dados especiais chamados dados de superioridade.

\textbf{Manobras.} Você aprende três manobras, à sua escolha, que são detalhadas
em ``Manobras'', a seguir.

Muitas manobras aprimoram um ataque de várias formas.  Você só pode usar uma
manobra por ataque.

Você aprende duas manobras adicionais, à sua escolha, no 7°, 10° e 15° nível. A
cada vez que você aprende uma nova manobra, você pode substituir uma manobra
conhecida por uma diferente.

\textbf{Dados de Superioridade.} Você tem quatro dados de superioridade, que são
d8s. Um dado de superioridade é gasto quando você usa-o. Você recupera todos os
dados de superioridade gastos quando terminar um descanso curto ou longo.

Você adquire outro dado de superioridade no 7° nível e mais um no 15° nível.

\textbf{Teste de Resistência.} Algumas das suas manobras exigem que o alvo
realize um teste de resistência contra o efeito da manobra. A CD do teste de
resistência é calculada a seguir:
\begin{center}
\textbf{CD para suas manobras} = 8 + bônus de proficiência + seu modificador de
Força ou Destreza (à sua escolha).
\end{center}

\subsubsection{Estudioso da Guerra}%

No 3° nível, você ganha proficiência com um tipo de ferramenta de artesão, à sua
escolha.

\subsubsection{Conheça seu Inimigo}%

A partir do 7° nível, se você gastar, pelo menos, 1 minuto observando ou
interagindo com outra criatura fora de combate, você pode aprender certas
informações sobre as capacidades dela comparadas as suas. O Mestre conta a você
se a criatura é igual, superior ou inferior a você a respeito de duas das
seguintes características, à sua escolha:

\begin{itemize}
    \item Valor de Força
    \item Valor de Destreza
    \item Valor de Constituição
    \item Classe de Armadura
    \item Pontos de Vida atuais
    \item Nível total de classe (se possuir)
    \item Níveis da classe guerreiro (se possuir)
\end{itemize}

\subsubsection{Superiorridade em Combate Aprimorada}%

No 10° nível, seus dados de superioridade se tornam d10s.

No 18° nível, eles se tornam d12s.

\subsubsection{Implacável}%

No 15º nível, quando você rolar iniciativa e não tiver nenhum dado de
superioridade restante, você recupera um dado de superioridade.

\subsubsection{Manobras}%

As manobras são apresentadas em ordem alfabética.

\paragraph{Aparar.} Quando outra criatura causar dano a você com um ataque
corpo-a-corpo, você pode usar sua reação e gastar um dado de superioridade para
reduzir o dano pelo número rolado no dado de superioridade + seu modificador de
Destreza.

\paragraph{Ataque Ameaçador.} Quando você atingir uma criatura com um ataque com
arma, você pode gastar um dado de superioridade para tentar amedrontar o alvo.
Você adiciona seu dado de superioridade a jogada de dano do ataque e o alvo deve
realizar um teste de resistência de Sabedoria. Se falhar, ele ficará com medo de
você até o final do seu próximo turno.

\paragraph{Ataque de Encontrão.} Quando você atingir uma criatura com um ataque
com arma, você pode gastar um dado de superioridade para tentar empurrar o alvo
para trás. Você adiciona seu dado de superioridade a jogada de dano do ataque e,
se o alvo for Grande ou menor, ele deve realizar um teste de resistência de
Força. Se falhar, você empurra o alvo para até 4,5 metros de você.

\paragraph{Ataque de Finta.} Você pode gastar um dado de superioridade e usar
uma ação bônus, no seu turno, para fintar, escolhendo uma criatura a 1,5 metro
de você como alvo. Você tem vantagem na sua próxima jogada de ataque contra essa
criatura, nesse turno. Se o ataque atingir, você adiciona seu dado de
superioridade ao dano do ataque.

\paragraph{Ataque de Manobra.} Quando você atingir uma criatura com um ataque
com arma, você pode gastar um dado de superioridade para tentar manobrar um de
seus companheiros para uma posição mais vantajosa. Você adiciona seu dado de
superioridade a jogada de dano do ataque e escolhe uma criatura amigável que
possa ver ou ouvir você. Aquela criatura pode usar sua reação para se mover até
metade do seu deslocamento, sem provocar ataques de oportunidade do alvo do seu
ataque.

\paragraph{Ataque de Precisão.} Quando você realizar uma jogada de ataque com
arma contra uma criatura, você pode gastar um dado de superioridade para
adicioná-lo a jogada. Você pode usar essa manobra antes ou depois de realizar a
jogada de ataque, mas deve usá-la antes de qualquer efeito do ataque ser
aplicado.

\paragraph{Ataque Desarmante.} Quando você atingir uma criatura com um ataque
com arma, você pode gastar um dado de superioridade para tentar desarmar o alvo,
forçando-o a derrubar um item, à sua escolha, que ele esteja empunhando. Você
adiciona o dado de superioridade a jogada de dano do ataque e o alvo deve
realizar um teste de resistência de Força. Se fracassar, ele derrubará o objeto
escolhido. O objeto cai aos pés dele.

\paragraph{Ataque Estendido.} Quando você atingir uma criatura com um ataque
corpo-a-corpo com arma, você pode gastar um dado de superioridade para aumentar
o alcance do seu ataque em 1,5 metro. Se você atingir, você adiciona o seu dado
de superioridade ao dano causado pelo ataque.

\paragraph{Ataque Provocante.} Quando você atingir uma criatura com um ataque
com arma, você pode gastar um dado de superioridade para tentar incitar a alvo a
atacar você. Você adiciona seu dado de superioridade a jogada de dano do ataque
e o alvo deve realizar um teste de resistência de Sabedoria. Se falhar, o alvo
terá desvantagem em todas as jogadas de ataque contra alvos diferentes de você,
até o fim do seu próximo turno.

\paragraph{Ataque Trespassante.} Quando você atingir uma criatura com um ataque
corpo-a-corpo com arma, você pode gastar um dado de superioridade para tentar
causar dano a outra criatura com o mesmo ataque. Escolha uma criatura a 1,5
metro do alvo original e que esteja no seu alcance. Se a jogada de ataque
original atingiria a segunda criatura, ela sofre dano igual ao número rolado no
dado de superioridade. O dano é do mesmo tipo que o causado pelo ataque
original.

\paragraph{Avaliação Tática} Quando você fizer um teste de Inteligência
(Investigação), Inteligência (História) ou Sabedoria (Intuição), você pode
gastar um dado de superioridade e adicioná-lo a esse teste.

\paragraph{Contra-Atacar.} Quando uma criatura atacar você com um ataque
corpo-a-corpo e errar, você pode usar sua reação e gastar um dado de
superioridade para realizar um ataque corpo-a-corpo com arma contra essa
criatura.  Se você atingir, você adiciona seu dado de superioridade a jogada de
dano do ataque.

\paragraph{Derrubar.} Quando você atingir uma criatura com um ataque com arma,
você pode gastar um dado de superioridade para tentar derrubar o alvo no chão.
Você adiciona seu dado de superioridade a jogada de dano do ataque e, se o alvo
for Grande ou menor, ele deve realizar um teste de resistência de Força. Se
falhar, o alvo ficará caído no chão.

\paragraph{Emboscada} Quando você realizar um teste de Destreza (Furtividade) ou
uma jogada de iniciativa, você pode gastar um dado de superioridade e adicionar
o valor do dado na rolagem, desde que você não esteja incapacitado.

\paragraph{Engodo} Quando você estiver a até 1,5m de uma criatura em seu turno,
você pode gastar um dado de superioridade e trocar de lugar com essa criatura,
desde que você gaste pelo menos 1,5m de movimento e a criatura seja voluntária e
não esteja incapacitada. Esse movimento não provoca ataques de oportunidade.
Jogue o dado de superioridade. Até o começo do seu próximo turno, você ganha um
bônus na CA igual ao número rolado.

\paragraph{Enganchar} Quando uma criatura que você possa ver se move dentro do
alcance que você possui com uma arma corpo-a-corpo que você está empunhando,
você pode usar sua reação para gastar um dado de superioridade e realizar um
ataque contra essa criatura, usando essa arma. Se o ataque acertar, adicione o
dado de superioridade à jogada de dano da arma.

\paragraph{Golpe Distrativo.} Quando você atingir uma criatura com um ataque com
arma, você pode gastar um dado de superioridade para tentar distrair a criatura,
abrindo uma brecha para um de seus aliados. Você adiciona seu dado de
superioridade a jogada de dano do ataque. A próxima jogada de ataque realizada
contra o alvo por uma criatura diferente de você, tem vantagem, se o ataque for
realizado antes do começo do seu próximo turno.

\paragraph{Golpe do Comandante.} Quando você realiza a ação de Ataque, no seu
turno, você pode desistir de um dos seus ataques e usar uma ação bônus para
direcionar o ataque de um dos seus companheiros. Quando você faz isso, escolha
uma criatura amigável que possa ver ou ouvir você e gaste um dado de
superioridade. Essa criatura pode, imediatamente, usar sua reação para realizar
um ataque com arma, adicionando seu dado de superioridade a jogada de dano do
ataque.

\paragraph{Inspirar.} No seu turno, você pode usar uma ação bônus e gastar um
dado de superioridade para reforçar a determinação dos seus companheiros. Quando
o fizer, escolha uma criatura amigável que possa ver ou ouvir você. Essa
criatura ganha uma quantidade de pontos de vida temporários igual a sua rolagem
de dado de superioridade + seu modificador de Carisma.

\paragraph{Passo Evasivo.} Quando você se mover, você pode gastar um dado de
superioridade, role o dado e adicione o número rolado a sua CA até você terminar
seu deslocamento.

\paragraph{Presença Dominante} Quando você realizar um teste de Carisma
(Intimidação, Performance ou Persuasão), você pode gastar um dado de
superioridade e adicionar o resultado dele a esse teste.

\paragraph{Golpe Imobilizador} Imediatamente após acertar uma criatura com um
ataque corpo-a-corpo em seu turno, você pode gastar um dado de superioridade e
então tentar agarrar o alvo como uma ação bônus (veja o Livro do Jogador para as
regras sobre Agarrar). Adicione o dado de superioridade ao seu teste de Força
(Atletismo).

\paragraph{Lançamento Rápido} Como uma ação bônus, você pode gastar um dado de
superioridade e realizar um ataque com uma arma que tenha a propriedade de
arremesso. Você pode sacar a arma como parte dessa ação de ataque. Se você
acertar, adicione o dado de superioridade na jogada de dano da arma.
\end{multicols}

%%%%%%%%%%%%%%%%%%%%%%%%%%%%%%%%%%%%%%%%%%%%%%%%%%%%%%%%%%%%%%%%%%%%%%%%%%%%%%%%
